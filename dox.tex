\documentclass[aspectratio=169]{beamer}
\usepackage{luatexja-fontspec}
\usepackage{fontspec-luatex}
\usepackage{unicode-math}
\usepackage{color}
\usetheme{metropolis}
\renewcommand{\kanjifamilydefault}{\gtdefault}% 既定をゴシック体に
\setmainjfont{Noto Sans CJK JP}
\setsansjfont{Noto Sans CJK JP}
\setmonojfont{Noto Sans CJK JP}
\title{差があるとはなにか}
\begin{document}
    \begin{frame}
        \titlepage
    \end{frame}

    \begin{frame}
        \frametitle{ABテストでよくあること}
        \begin{center}
        \begin{tabular}{ccc}
            \hline
            パターン & サンプル数 & コンバージョン数 \\
            original & 150 & 30 \\
            test & 200 & 55 \\
            \hline
        \end{tabular}
        \end{center}
        \begin{itemize}
            \item これは差があるのか?
            \item どうやって差を定義するのか?
        \end{itemize}
    \end{frame}

    \begin{frame}
        \frametitle{検定}
        \begin{itemize}
            \item 複数の集団の間に差があるかどうかを調べることを\textcolor{red}{検定}という
            \begin{block}{検定の方法}
                \begin{itemize}
                    \item 2群間の差がないと仮定する(\textcolor{red}{帰無仮説})
                    \item 帰無仮説から差があるとしか考えられない事実を導き出す
                    \begin{itemize}
                        \item 帰無仮説を\textcolor{red}{棄却}して、\textcolor{red}{対立仮説}を支持する
                    \end{itemize}
                \end{itemize}
            \end{block}
        \end{itemize}
    \end{frame}

    \begin{frame}
        \frametitle{検定と2種類の誤り}
            \begin{itemize}
                \item 帰無仮説$H_0$を設定する
                \item $H_0$が棄却されたとき対立仮説$H_1$を設定する
                \item $H_0$が棄却できるとき「\textcolor{red}{有意である}」といえる
            \end{itemize}
            \begin{block}{第1種の誤り}
                \begin{itemize}
                    \item $H_0$が正しい(成立している)のに棄却されること
                    \item この確率は$\alpha$と表される
                \end{itemize}
                
            \end{block}
            \begin{block}{第2種の誤り}
                \begin{itemize}
                    \item $H_0$が正しくない(成立していない)のに棄却されないこと
                    \item この確率は$\beta$と表される
                \end{itemize}
            \end{block}
    \end{frame}

    \begin{frame}
        \frametitle{ABテストの例}
        \begin{itemize}
            \item 帰無仮説$H_0$: originalとtestに差はない
            \item 対立仮説$H_1$: originalとtestに差がある
            \item 第1種の誤り: \textcolor{red}{originalとtestに差がないのに棄却すること}
            \begin{itemize}
                \item 差がなかったら棄却されるべきではない
            \end{itemize}
            \item 第2種の誤り: \textcolor{red}{originalとtestに差があるのに棄却されないこと}
            \begin{itemize}
                \item 差があったら棄却されるべき
            \end{itemize}
        \end{itemize}
    \end{frame}

    \begin{frame}
        \frametitle{イカサマコインの例}
        あるコインがイカサマかどうか?
        \begin{itemize}
            \item 帰無仮説$H_0$: コインはイカサマではない($P = 0.5$)
            \item 対立仮説$H_1$: コインはイカサマである($P \neq 0.5$)
        \end{itemize}
        $P$ コインが表を出す確率
        \vspace{1cm}
        \begin{block}{$H_0$の棄却方式(コインがイカサマであるとする方式)}
            \begin{itemize}
                \item 6回投げて、表が出た回数が0あるいは6回の場合に棄却する
            \end{itemize}
        \end{block}
    \end{frame}

    \begin{frame}
        \frametitle{第1種の誤りを犯す確率}
            $H_0$が成り立った下で、$H_0$を棄却する確率$\alpha$は
            \begin{equation}
                \alpha = Pr(x=6)+Pr(x=0) = 0.5^6 + 0.5^6 = 0.0313
            \end{equation}

            \begin{itemize}
                \item $\alpha$は一般的に小さい
            \end{itemize}
    \end{frame}

    \begin{frame}
        \frametitle{第2種の誤りを犯す確率}
            $H_1$が成り立った下で、$H_0$を棄却しない確率$\beta$を求める
            \begin{itemize}
                \item 棄却しない確率 = 1 - 棄却する確率
                \item $P = 0.6$(表が出る確率が0.6)とする
            \end{itemize}

            $H_1$が成り立った下で、$H_0$を棄却しない確率$\beta$は
            \begin{equation}
                \beta = 1 - (Pr(x=6)+Pr(x=0)) = 0.6^6 + 0.4^6 = 0.9492
            \end{equation}

            \begin{itemize}
                \item $\beta$は一般的に大きい
            \end{itemize}
    \end{frame}
\end{document}