\documentclass[aspectratio=169]{beamer}
\usepackage{luatexja-fontspec}
\usepackage{fontspec-luatex}
\usepackage{unicode-math}
\usepackage{color}
\usetheme{metropolis}
\renewcommand{\kanjifamilydefault}{\gtdefault}% 既定をゴシック体に
\setmainjfont{Noto Sans CJK JP}
\setsansjfont{Noto Sans CJK JP}
\setmonojfont{Noto Sans CJK JP}
\title{差があるとはなにか}
\begin{document}
    \begin{frame}
        \titlepage
    \end{frame}
    \begin{frame}
        \frametitle{ABテストでよくあること}
        \begin{center}
        \begin{tabular}{ccc}
            \hline
            パターン & サンプル数 & コンバージョン数 \\
            original & 150 & 30 \\
            test & 200 & 55 \\
            \hline
        \end{tabular}
        \end{center}
        \begin{itemize}
            \item これは差があるのか?
            \item どうやって差を定義するのか?
        \end{itemize}
    \end{frame}
    \begin{frame}
        \frametitle{検定}
        \begin{itemize}
            \item 複数の集団の間に差があるかどうかを調べることを\textcolor{red}{検定}という
            \begin{block}{検定の方法}
                \begin{itemize}
                    \item 2群間の差がないと仮定する(\textcolor{red}{帰無仮説})
                    \item 帰無仮説から差があるとしか考えられない事実を導き出す
                    \begin{itemize}
                        \item 帰無仮説を\textcolor{red}{棄却}して、\textcolor{red}{対立仮説}を支持する
                    \end{itemize}
                \end{itemize}
            \end{block}
        \end{itemize}
    \end{frame}
\end{document}